\documentclass[presentation]{beamer}

\usetheme{Warsaw}
\author{Alex Rice}

\title{Biased whisker-based composition in higher categories}
\date[CaCS 2021]{Categories and Companions Symposium, June 2021}

\usepackage{quiver}
\newcommand{\comp}{\star}

\setbeamertemplate{page number in head/foot}[totalframenumber]
\setbeamertemplate{navigation symbols}{}

\begin{document}

{
  \setbeamertemplate{footline}{}
  \setbeamertemplate{headline}{} %
  \begin{frame}[noframenumbering]
\maketitle
\end{frame}}

\begin{frame}{Outline}
  \tableofcontents
\end{frame}

\section{Globular sets}

\begin{frame}
  \frametitle{What do we mean by a higher category?}
  A regular (1-)category consists of objects and arrows.

  In higher category theory we expand this to allow arrows of higher dimensions between lower dimensional arrows.

  What should these higher dimensional arrows look like?
\end{frame}

\begin{frame}
  \frametitle{Globular Sets}
  Globular sets are one natural shape of higher categories:

  \begin{columns}
    \begin{column}{0.5\textwidth}
      \pause
      \begin{block}
        {Ordinary 1-categories}
        % https://q.uiver.app/?q=WzAsMixbMCwxLCJDXzAiXSxbMCwwLCJDXzEiXSxbMSwwLCJzIiwyLHsib2Zmc2V0IjoyfV0sWzEsMCwidCIsMCx7Im9mZnNldCI6LTJ9XV0=
        \[\begin{tikzcd}
            {C_1} \\
            {C_0}
            \arrow["s"', shift right=2, from=1-1, to=2-1]
            \arrow["t", shift left=2, from=1-1, to=2-1]
          \end{tikzcd}\]
      \end{block}
    \end{column}
    \pause
    \begin{column}{0.5\textwidth}
      \begin{block}
        {Globular sets}
        % https://q.uiver.app/?q=WzAsNCxbMCwzLCJHXzAiXSxbMCwyLCJHXzEiXSxbMCwxLCJHXzIiXSxbMCwwXSxbMSwwLCJzXzAiLDIseyJvZmZzZXQiOjJ9XSxbMSwwLCJ0XzAiLDAseyJvZmZzZXQiOi0yfV0sWzIsMSwic18xIiwyLHsib2Zmc2V0IjoyfV0sWzIsMSwidF8xIiwwLHsib2Zmc2V0IjotMn1dLFszLDIsIiIsMCx7InN0eWxlIjp7ImJvZHkiOnsibmFtZSI6ImRvdHRlZCJ9LCJoZWFkIjp7Im5hbWUiOiJub25lIn19fV1d
        \[\begin{tikzcd}
            {} \\
            {G_2} \\
            {G_1} \\
            {G_0}
            \arrow["{s_0}"', shift right=2, from=3-1, to=4-1]
            \arrow["{t_0}", shift left=2, from=3-1, to=4-1]
            \arrow["{s_1}"', shift right=2, from=2-1, to=3-1]
            \arrow["{t_1}", shift left=2, from=2-1, to=3-1]
            \arrow[dotted, no head, from=1-1, to=2-1]
          \end{tikzcd}\]
      \end{block}
    \end{column}
  \end{columns}
\end{frame}

\begin{frame}
  \frametitle{Globular Sets}

  Globular sets can be defined in a few different ways.

  \pause{}

  \begin{definition}
    A \emph{globular set} \(\mathcal{G}\) consists of sets \(G_n\) for each \(n\) and maps \(s_n,t_n : G_{n+1} \to G_n\) for each \(n\) such that the following \emph{globularity conditions} hold:
    \begin{align*}
      s_n \circ s_{n+1} &= s_n \circ t_{n+1}\\
      t_n \circ s_{n+1} &= t_n \circ t_{n+1}
    \end{align*}
  \end{definition}
  \pause{}
  We will write \(f : x \to y\) for an object \(f\) in \(G_{n+1}\) with \(s(f) = x\) and \(t(f) = y\).

  \pause{}
  Let the objects of the globular set be it's \(0\)-cells, morphisms between these be \(1\)-cells, \(\dots\)

\end{frame}

\begin{frame}
  \frametitle{Examples}
  \begin{itemize}
  \item Ordinary 1-categories
  \item 2-categories, 3-categories, \(\dots\)
  \item Monoidal categories, braided monoidal categories, \(\dots\)
  \item Martin L\"of Type Theory/non-Cubical Homotopy Type Theory
  \end{itemize}
\end{frame}

\section{Composition in Globular sets}

\begin{frame}
  \frametitle{Unbiased vs Biased Composition}
  Common definitions of categories have a biased composition.
  \pause{}

  Could we define a category with a ternary composition instead? Unary composition? Nullary composition?

  \pause{}
  \begin{block}{Unbiased vs Biased}
    In an \emph{unbiased} definition, we allow every possible composition operation.

    In a \emph{biased} definition, we only allow a subset of these operations.
  \end{block}
\end{frame}

\begin{frame}[fragile]
  \frametitle{Composition in infinity categories}
  \begin{block}{Composition of 2 cells}
    Composition along a 1-boundary:
    \begin{tikzcd}
      \bullet \arrow[r, bend right=90, ""{name=S1}] \arrow[r, ""{name=T1,below}, ""{name=S2}] \arrow[r, bend left=90, ""{name=T2, below}] & \bullet
      \arrow[Rightarrow, "\alpha", from=S1, to=T1]
      \arrow[Rightarrow, "\beta", from=S2, to=T2]
    \end{tikzcd}
    \pause{}

    Codimension along a 0-boundary:
    \begin{tikzcd}
      \bullet \arrow[r, bend right=49, ""{name=S1}] \arrow[r, bend left=49, ""{name=T1, below}] & \bullet \arrow[r, bend right=49, ""{name=S2}] \arrow[r, bend left=49, ""{name=T2, below}] & \bullet
      \arrow[Rightarrow, "\alpha", from=S1, to=T1]
      \arrow[Rightarrow, "\beta", from=S2, to=T2]
    \end{tikzcd}
  \end{block}
\end{frame}
\begin{frame}[fragile]
  \frametitle{Higher compositions and pasting diagrams}
  \begin{columns}
    \begin{column}{0.5\textwidth}
      \begin{exampleblock}{3-cell composition}
        % https://q.uiver.app/?q=WzAsMixbMiwwLCJcXGJ1bGxldCJdLFswLDAsIlxcYnVsbGV0Il0sWzEsMCwiIiwwLHsiY3VydmUiOjR9XSxbMSwwLCIiLDIseyJjdXJ2ZSI6LTR9XSxbMiwzLCIiLDAseyJvZmZzZXQiOi01LCJzaG9ydGVuIjp7InNvdXJjZSI6MjAsInRhcmdldCI6MjB9fV0sWzIsMywiIiwyLHsib2Zmc2V0Ijo1LCJzaG9ydGVuIjp7InNvdXJjZSI6MjAsInRhcmdldCI6MjB9fV0sWzIsMywiIiwxLHsic2hvcnRlbiI6eyJzb3VyY2UiOjIwLCJ0YXJnZXQiOjIwfX1dLFs0LDYsIiIsMCx7InNob3J0ZW4iOnsic291cmNlIjoyMCwidGFyZ2V0IjoyMH19XSxbNiw1LCIiLDAseyJzaG9ydGVuIjp7InNvdXJjZSI6MjAsInRhcmdldCI6MjB9fV1d
\[\begin{tikzcd}
	\bullet && \bullet
	\arrow[""{name=0, anchor=center, inner sep=0}, curve={height=24pt}, from=1-1, to=1-3]
	\arrow[""{name=1, anchor=center, inner sep=0}, curve={height=-24pt}, from=1-1, to=1-3]
	\arrow[""{name=2, anchor=center, inner sep=0}, shift left=5, shorten <=6pt, shorten >=6pt, Rightarrow, from=0, to=1]
	\arrow[""{name=3, anchor=center, inner sep=0}, shift right=5, shorten <=6pt, shorten >=6pt, Rightarrow, from=0, to=1]
	\arrow[""{name=4, anchor=center, inner sep=0}, shorten <=6pt, shorten >=6pt, Rightarrow, from=0, to=1]
	\arrow[shorten <=2pt, shorten >=2pt, from=2, to=4]
	\arrow[shorten <=2pt, shorten >=2pt, from=4, to=3]
\end{tikzcd}\]
% https://q.uiver.app/?q=WzAsMixbMCwwLCJcXGJ1bGxldCJdLFsyLDAsIlxcYnVsbGV0Il0sWzAsMSwiIiwwLHsiY3VydmUiOi01fV0sWzAsMSwiIiwyLHsiY3VydmUiOjV9XSxbMCwxXSxbNCwyLCIiLDEseyJvZmZzZXQiOi00LCJzaG9ydGVuIjp7InNvdXJjZSI6MjAsInRhcmdldCI6MjB9fV0sWzQsMiwiIiwxLHsib2Zmc2V0Ijo0LCJzaG9ydGVuIjp7InNvdXJjZSI6MjAsInRhcmdldCI6MjB9fV0sWzMsNCwiIiwxLHsib2Zmc2V0IjotNCwic2hvcnRlbiI6eyJzb3VyY2UiOjIwLCJ0YXJnZXQiOjIwfX1dLFszLDQsIiIsMSx7Im9mZnNldCI6NCwic2hvcnRlbiI6eyJzb3VyY2UiOjIwLCJ0YXJnZXQiOjIwfX1dLFs1LDYsIiIsMSx7InNob3J0ZW4iOnsic291cmNlIjoyMCwidGFyZ2V0IjoyMH19XSxbNyw4LCIiLDEseyJzaG9ydGVuIjp7InNvdXJjZSI6MjAsInRhcmdldCI6MjB9fV1d
\[\begin{tikzcd}
	\bullet && \bullet
	\arrow[""{name=0, anchor=center, inner sep=0}, curve={height=-30pt}, from=1-1, to=1-3]
	\arrow[""{name=1, anchor=center, inner sep=0}, curve={height=30pt}, from=1-1, to=1-3]
	\arrow[""{name=2, anchor=center, inner sep=0}, from=1-1, to=1-3]
	\arrow[""{name=3, anchor=center, inner sep=0}, shift left=4, shorten <=4pt, shorten >=4pt, Rightarrow, from=2, to=0]
	\arrow[""{name=4, anchor=center, inner sep=0}, shift right=4, shorten <=4pt, shorten >=4pt, Rightarrow, from=2, to=0]
	\arrow[""{name=5, anchor=center, inner sep=0}, shift left=4, shorten <=4pt, shorten >=4pt, Rightarrow, from=1, to=2]
	\arrow[""{name=6, anchor=center, inner sep=0}, shift right=4, shorten <=4pt, shorten >=4pt, Rightarrow, from=1, to=2]
	\arrow[shorten <=3pt, shorten >=3pt, from=3, to=4]
	\arrow[shorten <=3pt, shorten >=3pt, from=5, to=6]
      \end{tikzcd}\]
    % https://q.uiver.app/?q=WzAsMyxbMCwwLCJcXGJ1bGxldCJdLFsxLDAsIlxcYnVsbGV0Il0sWzIsMCwiXFxidWxsZXQiXSxbMCwxLCIiLDAseyJjdXJ2ZSI6LTJ9XSxbMCwxLCIiLDIseyJjdXJ2ZSI6Mn1dLFsxLDIsIiIsMix7ImN1cnZlIjotMn1dLFsxLDIsIiIsMix7ImN1cnZlIjoyfV0sWzQsMywiIiwyLHsib2Zmc2V0IjotMywic2hvcnRlbiI6eyJzb3VyY2UiOjIwLCJ0YXJnZXQiOjIwfX1dLFs0LDMsIiIsMCx7Im9mZnNldCI6Mywic2hvcnRlbiI6eyJzb3VyY2UiOjIwLCJ0YXJnZXQiOjIwfX1dLFs2LDUsIiIsMix7Im9mZnNldCI6LTMsInNob3J0ZW4iOnsic291cmNlIjoyMCwidGFyZ2V0IjoyMH19XSxbNiw1LCIiLDAseyJvZmZzZXQiOjMsInNob3J0ZW4iOnsic291cmNlIjoyMCwidGFyZ2V0IjoyMH19XSxbNyw4LCIiLDIseyJzaG9ydGVuIjp7InNvdXJjZSI6MjAsInRhcmdldCI6MjB9fV0sWzksMTAsIiIsMix7InNob3J0ZW4iOnsic291cmNlIjoyMCwidGFyZ2V0IjoyMH19XV0=
\[\begin{tikzcd}
	\bullet & \bullet & \bullet
	\arrow[""{name=0, anchor=center, inner sep=0}, curve={height=-12pt}, from=1-1, to=1-2]
	\arrow[""{name=1, anchor=center, inner sep=0}, curve={height=12pt}, from=1-1, to=1-2]
	\arrow[""{name=2, anchor=center, inner sep=0}, curve={height=-12pt}, from=1-2, to=1-3]
	\arrow[""{name=3, anchor=center, inner sep=0}, curve={height=12pt}, from=1-2, to=1-3]
	\arrow[""{name=4, anchor=center, inner sep=0}, shift left=3, shorten <=3pt, shorten >=3pt, Rightarrow, from=1, to=0]
	\arrow[""{name=5, anchor=center, inner sep=0}, shift right=3, shorten <=3pt, shorten >=3pt, Rightarrow, from=1, to=0]
	\arrow[""{name=6, anchor=center, inner sep=0}, shift left=3, shorten <=3pt, shorten >=3pt, Rightarrow, from=3, to=2]
	\arrow[""{name=7, anchor=center, inner sep=0}, shift right=3, shorten <=3pt, shorten >=3pt, Rightarrow, from=3, to=2]
	\arrow[shorten <=2pt, shorten >=2pt, from=4, to=5]
	\arrow[shorten <=2pt, shorten >=2pt, from=6, to=7]
\end{tikzcd}\]
\end{exampleblock}

\end{column}
\begin{column}{0.5\textwidth}
  \begin{exampleblock}{}
    % https://q.uiver.app/?q=WzAsMyxbMCwwLCJcXGJ1bGxldCJdLFsxLDAsIlxcYnVsbGV0Il0sWzIsMCwiXFxidWxsZXQiXSxbMCwxLCJmIl0sWzEsMiwiIiwwLHsiY3VydmUiOi0yfV0sWzEsMiwiIiwwLHsiY3VydmUiOjJ9XSxbNSw0LCJcXGFscGhhIiwwLHsic2hvcnRlbiI6eyJzb3VyY2UiOjIwLCJ0YXJnZXQiOjIwfX1dXQ==
\[\begin{tikzcd}
	\bullet & \bullet & \bullet
	\arrow["f", from=1-1, to=1-2]
	\arrow[""{name=0, anchor=center, inner sep=0}, curve={height=-12pt}, from=1-2, to=1-3]
	\arrow[""{name=1, anchor=center, inner sep=0}, curve={height=12pt}, from=1-2, to=1-3]
	\arrow["\alpha", shorten <=3pt, shorten >=3pt, Rightarrow, from=1, to=0]
      \end{tikzcd}\]
    % https://q.uiver.app/?q=WzAsMyxbMCwwLCJcXGJ1bGxldCJdLFsxLDAsIlxcYnVsbGV0Il0sWzMsMCwiXFxidWxsZXQiXSxbMCwxLCIiLDIseyJjdXJ2ZSI6Mn1dLFswLDEsIiIsMCx7ImN1cnZlIjotMn1dLFsxLDIsIiIsMix7ImN1cnZlIjotNX1dLFsxLDIsIiIsMix7ImN1cnZlIjo1fV0sWzEsMiwiIiwyLHsiY3VydmUiOi0zfV0sWzEsMiwiIiwyLHsiY3VydmUiOjN9XSxbMyw0LCIiLDIseyJzaG9ydGVuIjp7InNvdXJjZSI6MjAsInRhcmdldCI6MjB9fV0sWzgsNywiIiwyLHsib2Zmc2V0IjotNSwic2hvcnRlbiI6eyJzb3VyY2UiOjIwLCJ0YXJnZXQiOjIwfX1dLFs4LDcsIiIsMCx7Im9mZnNldCI6NSwic2hvcnRlbiI6eyJzb3VyY2UiOjIwLCJ0YXJnZXQiOjIwfX1dLFs3LDUsIiIsMix7InNob3J0ZW4iOnsic291cmNlIjoyMCwidGFyZ2V0IjoyMH19XSxbNiw4LCIiLDIseyJzaG9ydGVuIjp7InNvdXJjZSI6MjAsInRhcmdldCI6MjB9fV0sWzEwLDExLCIiLDIseyJzaG9ydGVuIjp7InNvdXJjZSI6MjAsInRhcmdldCI6MjB9fV1d
\[\begin{tikzcd}
	\bullet & \bullet && \bullet
	\arrow[""{name=0, anchor=center, inner sep=0}, curve={height=12pt}, from=1-1, to=1-2]
	\arrow[""{name=1, anchor=center, inner sep=0}, curve={height=-12pt}, from=1-1, to=1-2]
	\arrow[""{name=2, anchor=center, inner sep=0}, curve={height=-40pt}, from=1-2, to=1-4]
	\arrow[""{name=3, anchor=center, inner sep=0}, curve={height=40pt}, from=1-2, to=1-4]
	\arrow[""{name=4, anchor=center, inner sep=0}, curve={height=-18pt}, from=1-2, to=1-4]
	\arrow[""{name=5, anchor=center, inner sep=0}, curve={height=18pt}, from=1-2, to=1-4]
	\arrow[shorten <=3pt, shorten >=3pt, Rightarrow, from=0, to=1]
	\arrow[""{name=6, anchor=center, inner sep=0}, shift left=5, shorten <=5pt, shorten >=5pt, Rightarrow, from=5, to=4]
	\arrow[""{name=7, anchor=center, inner sep=0}, shift right=5, shorten <=5pt, shorten >=5pt, Rightarrow, from=5, to=4]
	\arrow[shorten <=2pt, shorten >=2pt, Rightarrow, from=4, to=2]
	\arrow[shorten <=2pt, shorten >=2pt, Rightarrow, from=3, to=5]
	\arrow[shorten <=4pt, shorten >=4pt, from=6, to=7]
\end{tikzcd}\]
  \end{exampleblock}
\end{column}
  \end{columns}
  \end{frame}

\section{Whisker-based composition}

\begin{frame}
  \frametitle{Stable Compositions}
  \begin{definition}
    A \emph{Stable} binary composition is a composition of an \(n\)-cell \(a\) and an \(m\)-cell \(b\) along their \((\min(n,m)- 1)\)-boundary. We write this composition \(a \cdot_{k} b\) where \(k\) is the boundary dimension.
  \end{definition}
  \vspace{50em}
\end{frame}

\begin{frame}
  \frametitle{Whisker-based composition scheme}
  \begin{block}{Claim}
    A definition of \(\infty\)-categories which only allows stable compositions is valid and equivalent to a fully unbiased definition.
  \end{block}
  \pause{}
  \begin{theorem}
    Every pasting diagram can be realised as a tree of stable binary composites. Furthermore this realisation respects source and target maps.
  \end{theorem}
\end{frame}

\begin{frame}
  \frametitle{Inductive Characterisation of Pasting Diagrams}

  Pasting diagrams ``with a focus'' are uniquely generated by the following rules.
  \begin{itemize}
  \item The singleton pasting diagram \(x\), is a pasting diagram with focus \(x\).
  \item If \(\Gamma\) is a pasting diagram with focus \(x\), then \(\Gamma, y, f : x \to y\) is a pasting diagram with focus \(f\).
  \item If \(\Gamma\) is a pasting diagram with focus \(f : x \to y\), then it is also a pasting diagram with focus \(y\).
  \end{itemize}
  A pasting diagram is a pasting diagram with a 0-dimensional focus.
\end{frame}

\begin{frame}
  \frametitle{Proof Sketch}
  \begin{columns}
    \begin{column}{0.5\textwidth}
      \begin{definition}
        If \(a\) is a cell and \(x\) is a variable, the \emph{principle replacement} \(a\langle x \rangle\) is given recursively by:
        \begin{itemize}
        \item If \(a = b \cdot c\) then \(a \langle x \rangle = b \cdot (c\langle x \rangle)\).
        \item If \(a\) is a variable then \(a\langle x \rangle = x\).
        \end{itemize}
      \end{definition}
    \end{column}
    \pause{}
    \begin{column}{0.5\textwidth}
      \begin{definition}
      For pasting diagram \(\Gamma\), define its \emph{stabilised form} \(S(\Gamma)\) by induction:
      \begin{itemize}
      \item If \(\Gamma\) is a singleton \(x\), then \(S(\Gamma) = x\).
      \item If \(\Gamma = \Delta, y, f\) and \(\dim(f) > \dim(\Delta)\) then \(S(\Gamma) = S(\Delta)\langle f \rangle\).
      \item If \(\Gamma = \Delta, y, f\) and \(\dim(f) \leq \dim(\Delta)\) then \(S(\Gamma) = S(\Delta) \cdot (\delta^+_{\dim(y)}(S(\Delta)))\langle f \rangle \).
      \end{itemize}
    \end{definition}
    \end{column}
  \end{columns}
\end{frame}
\begin{frame}
  \frametitle{Example}

\end{frame}

\begin{frame}
  \frametitle{Conclusions}
  \begin{itemize}
  \item We introduced the notion of a stable composite.
  \item We define a translation \(S\) from an arbitrary pasting diagram to a tree of stable binary composites.
  \item This translation \(S\) respects boundaries.
  \item Future aims include:
    \begin{itemize}
    \item Proving the existence of an equivalence.
    \item Generalise the work to prove the viability of many composition schemes.
    \end{itemize}
  \end{itemize}
\end{frame}



\end{document}